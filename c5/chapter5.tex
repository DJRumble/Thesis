\chapter{Temperature mapping}
\label{ch:chapter5}

\section{Introduction to the temperature equation}
\subsection{Submillimetre Flux ratio}
%MWC 297
Using the ratio of 450\,$\micron$ and 850\,$\micron$ fluxes from SCUBA-2, 
we develop a method that utilises the two frequency observations of the same 
region where the ratio depends partly on the dust temperature ($T_{\mathrm{d}}$) 
via the Planck function and also on the dust opacity spectral index, $\beta$ 
(a dimensionless term dependent on the grain model as proposed by 
\citeauthor{Hildebrand:1983fy}\,\citeyear{Hildebrand:1983fy}), as described by 
\begin{equation}
\frac{S_{\mathrm{450}}}{S_{\mathrm{850}}} = \left ( \frac{850}{450} \right )^{3+\beta }\left ( \frac{\exp(hc/\lambda _{\mathrm{850}}k_{\mathrm{B}}T_{\mathrm{d}})-1}{\exp(hc/\lambda _{\mathrm{450}}k_{\mathrm{B}}T_{\mathrm{d}})-1} \right ), \label{eqn:temp}
\end{equation}
otherwise referred to as `the temperature equation' \citep{Reid:2005ly}.
%MWC 297
There is no analytical solution for temperature and so pixel values are inferred from 
a lookup table. The method by which temperature maps are made can be split into 
two distinct parts: creating maps of flux ratio from input 450\,$\micron$ and 850\,$\micron$ 
data and building temperature maps based on the ratio maps. Both methods were 
discussed by \cite{Hatchell:2013ij}, for here on referred to as the H13 method. 
We focus on the development of this method and the additional features that have 
been incorporated.

\subsection{Beta}
%MWC 297
\begin{figure}
\begin{centering}
\includegraphics[scale=0.35]{/Users/damian/Documents/Thesis_et_al/images/20131228_FluxRatio.pdf}
\caption{Flux ratio as a function of temperature as described by Equation \ref{eqn:temp}. The temperature range is that commonly observed in protostellar cores.} \label{fig:SR_temp}
\end{centering}
\end{figure} 
%MWC 297
Temperature is known to influence the process by which dust grains coagulate 
and form icy mantles and therefore the value of $\beta$. Observations by 
\cite{Ubach:2012fk} have shown decreases in $\beta$ in protoplanetary disks 
but for the most part there is little evidence that $\beta$ changes significantly in 
pre/protostellar cores \citep{Schnee:2014uq}. \cite{Sadavoy:2013qf} fitted 
\emph{Herschel} 160\,$\micron$ to 500\,$\micron$ data with SCUBA-2 data in 
the Perseus B1 region  and concluded that $\beta$ is approximately 2.0 in 
extended, filamentary regions whereas it takes a lower value of approximately 
1.6 towards dense protostellar cores. 
%MWC 297
Figure~\ref{fig:SR_temp} describes how small changes in $\beta$ lead to a 
large range of flux ratios, especially at higher temperatures. For ratios of 3, 7 
and 9, a $\beta$ of 1.6 would return temperatures of 8.9, 25.4 and 85\,K whereas 
a $\beta$ of 2.0 would return temperatures of 7.6, 15.7 and 25\,K. Higher ratios 
indicate heating above that available from the Interstellar Radiation Field (ISRF) 
for any reasonable value of $\beta$. 
%MWC 297
Removing the requirement for the uncertainty in $\beta$ requires data at additional 
wavelengths, for example 250\,$\micron$ and 350\,$\micron$ as observed by 
\emph{Herschel}. Reconciling the angular scales of \emph{Herschel} observations 
with those of SCUBA-2 is a non-trivial process and goes beyond the scope of this paper. 
%MWC 297
Smaller values of $\beta$ are found to be consistent with grain growth which only 
occurs sufficiently close to compact structures \citep{Ossenkopf:1994vn}. 
\cite{Stutz:2010hq} used the dominance of extended structure to that of compact 
structure to argue for a uniform, higher value of $\beta$. Likewise \cite{Hatchell:2013ij} 
assumed a constant $\beta$, arguing that variation in temperature dominates to that 
of $\beta$ in NGC1333. On this basis we adopt a uniform $\beta$ of 1.8, a value 
consistent with the popular OH5 dust model proposed by \cite{Ossenkopf:1994vn} 
and studies of dense cores with \emph{Planck}, \emph{Herschel} and SCUBA-2 
\citep{Stutz:2010hq, Juvela:2011ys, Sadavoy:2013qf}. We note that in this regime an 
apparent fall in temperature towards the centre of a core might be symptomatic of 
low $\beta$ values and therefore we cannot be as certain about the temperatures 
at these points.


\subsection{Alpha}
\subsection{Temperature}



\section{Dual beam method}
\subsection{The JCMT beam (primary and secondary)}

%FROM MWC 297
The JCMT beam can be modelled as two Gaussian components \citep{Drabek:2012uq, 
Dempsey:2013uq}. The primary (or main) beam contains the bulk of the signal and is 
well described by a Gaussian, $G_{\mathrm{MB}}$, but in addition to this there is also 
a secondary beam which is much wider and lower in amplitude, $G_{\mathrm{SB}}$. 
Together they make up the 2-component beam of the telescope, 
%FROM MWC 297
\begin{equation}
G_{\mathrm{total}}=a G_{\mathrm{MB}} + b G_{\mathrm{SB}}, \label{eqn:effbeam} 
\end{equation}
where $a$ and $b$ are relative amplitude, listed in Table~\ref{table:beams} 
alongside the FWHM, $\theta$, of the primary (MB) and secondary (SB) beams.   
%FROM MWC 297
%\begin{table}
%\caption{JCMT beam properties}
%\label{table:beams}
%\begin{center}
%\centering
%\begin{tabular}{l | p{1cm} | p{1cm}  }
%		&	450\,$\micron$ &	850\,$\micron$ \\
%	\hline
%	$\theta_{\mathrm{MB}}$ &	7.9\arcsec\	&	13.0\arcsec\		\\
%	$\theta_{\mathrm{SB}}$	&	25.0\arcsec\	&	48.0\arcsec\		\\	
%	$a$	&	0.94	&	0.98		\\
%	$b$	&	0.06	&	0.02		\\
%	Pixel size&	4\arcsec\	&	6\arcsec\		\\
%	\hline
%\end{tabular}
%\end{center}
%JCMT beam Full Width Half Maximum ($\theta$) and relative amplitudes from \cite{Dempsey:2013uq} Table 1. Pixel sizes are those chosen by the JCMT SGBS data reduction team.

\subsection{Gaussian beam convolution}
%MWC 297
We introduce a secondary beam component into the H13 method, which previously 
assumed that the secondary component was negligible. This adds complexity 
to the convolution process as it requires convolution of the data with a normalised 
Gaussian of the form of the JCMT beam's primary and secondary components for 
the alternative wavelength. The primary component at 850\,\micron\ is then scaled with 
%MWC 297
\begin{equation}
\frac{a_{\mathrm{450}}\theta _{\mathrm{MB_{\mathrm{450}}}}^{2}}{a _{\mathrm{450}}\theta _{\mathrm{MB_{\mathrm{450}}}}^{2}+b _{\mathrm{450}}\theta _{\mathrm{SB_{\mathrm{450}}}}^{2}}, \label{eqn:scaleMB}
\end{equation}
%MWC 297
and likewise 
%MWC 297
\begin{equation}
\frac{b_{\mathrm{450}}\theta _{\mathrm{SB_{\mathrm{450}}}}^{2}}{a_{\mathrm{450}}\theta _{\mathrm{MB_{\mathrm{450}}}}^{2}+b_{\mathrm{450}}\theta _{\mathrm{SB_{\mathrm{450}}}}^{2}}, \label{eqn:scaleSB}
\end{equation} 
for the secondary component. The 450\,$\micron$ map is convolved with the 850\,$\micron$ 
beam is a similar way. Corresponding parts are then summed together for 450\,$\micron$ and 
850\,$\micron$ data separately to construct the convolved maps with an effective beam size of 
19.9\arcsec\ as shown in Figure~\ref{fig:mwc}.


\subsection{4 component dual beam convolution method}
%MWC 297
\begin{figure(}
\begin{center}
\includegraphics[scale=0.5]{/Users/damian/Documents/Thesis_et_al/images/temp_flowchart_simple.pdf}
\caption{A flow chart providing a simplified description of the ratio mapping process}
\label{fig:flow}
\end{center}
\end{figure*}

%MWC 297
Free parameters of our method are limited to $\beta$ (which we set at 1.8). Input 
450\,$\micron$ and 850\,$\micron$ flux density data (scaled in Jy/pixel) have fixed 
noise levels. Other fixed parameters which are used in the beam convolution 
include: the pixel area per  map, FWHM of the primary ($\theta_{\mathrm{MB}}$) 
and secondary ($\theta_{\mathrm{SB}}$) beams and beam amplitudes all of which 
are measured by \cite{Dempsey:2013uq} and given in Table \ref{table:beams}.
%MWC 297
Input maps are first convolved with the JCMT beam (Equation 1) at the alternate 
wavelength to match resolution. Pixel size is taken into account in this process.
The 450\,$\micron$ fluxes are then regridded onto the 850\,$\micron$ pixel grid. 
Data are then masked leaving only 5$\sigma$ detections or higher. 450\,$\micron$ 
fluxes are then divided by 850\,$\micron$ fluxes to create a map of flux ratio.
%MWC 297
Whereas the H13 method made a noise cut based on the variance array calculated 
during data reduction, our model introduces a cut based on a single noise estimate, 
following the method introduced by \cite{Salji:2013kx}. The data are masked to remove 
pixels which carry astronomical signal. The remaining pixels are placed in a histogram 
of intensity and a Gaussian is fitted to the distribution, from which a standard deviation, 
$\sigma$,  can be extracted as the noise level. This calculation is a robust form of 
measuring statistical noise that includes residual sky fluctuations.
%MWC 297
%TJW calculated flux in units per beam beam which necessititated a calibration factor and an associated error of 5 to 10per cent depending on wavelength \citep{Dempsey:2013uq} however since then GBS Internal Release 1 standard has revised the units of input data array to units per pixel which eliminates this necessary calbriation and source of statisical error. As a result the improved error derived from the noise level can be propagated through the various process with the model to produce new error arrays. 
%MWC 297
The inclusion of the secondary beam was found to decrease temperatures by between 5 per cent and 
9 per cent with the coldest regions experiencing the largest drop in temperature and warmest the least. 
%MWC 297
Applying a 5$\sigma$ cut based on the original 450\,$\micron$ data to mask uncertain regions of 
large scale structure after the beam convolution can lead to spuriously high values around the 
edges of our maps where fluxes from pixels below the threshold are contributing to those above, 
producing false positives. These `edge effects' are mitigated by clipping but we advise that 
where the highest temperature pixels meet the map edges these data be regarded with a degree of scepticism. 
%MWC 297
%Our model produces more reliable temperatures as a result of using the full beam whereas 
%the H13 method does not. For point sources, such as YSOs and prestellar cores, our results 
%show that the effect of the secondary beam is non-negligible, having an ameliorating effect 
%on ratio and temperature. Flux from individual pixels is spread out over a wider area through 
%convolution. At the 5$\sigma$ noise level we observe how unreliable data can encroach on 
%reliable data and produce artefacts known as `edge effects'. The offset in values of $\alpha$ 
%and $\beta$ between 450\,$\micron$ and 850\,$\micron$ listed in Table~\ref{table:beams} 
%produces contrasting beam profiles. Where underlying large scale structure encroaches on 
%significant data, we see a substantial divergence in fluxes in favour of exceptionally larger flux 
%ratios, resulting unrealistically high temperatures in these regions. 
%MWC 297
%the impact of these artefacts can be reduced through careful clipping of the data. Large scale structure regions 
%are inherently less dense than point-like dense cores and therefore often fall bellow the 5$\sigma$ noise level 
%and are cut from the maps. As these regions do not hold significant observations, it is not a concern that they 
%also contain artefacts. We cannot rule out all instances with this method as much of the SCUBA-2 data covers regions 
%of significant large-scale star-forming clouds. We cannot further adapt our processes to mitigate these effects, 
%however as this paper mainly studies YSOs which are point-like, the effect of these artefacts are unlikely to encroach 
%in these cases and we proceed with the proviso that a degree of scepticism should be cast towards 
%high temperature data on the edges of large scale regions. 
%MWC 297
Ratio maps are converted to temperature maps using Equation \ref{eqn:temp} 
implemented as a look-up table as there is no analytical solution. The H13 method 
subsequently cuts pixels with an arbitrary uncertainty in temperature of greater 
than 5.5\,K. We replace this with a cut of pixels of an uncertainty in temperature 
(calculated from the noise level propagated through the method described in 
Section 3.1) of greater than 5 per cent. 

\section{Propagation of error}
\subsection{Analytical calculation}
\subsection{Monte-Carlo method}
\section{Testing Dual beam method}
\section{Kernel method}
\subsection{Convolution kernel}
\subsection{Beam model/beam map}
\subsection{Preparing the Kernel}
\section{Propagation of error}
\subsection{Monte-Carlo method}
\section{Testing Kernel method}
\subsection{Pixel size problem}
\section{Comparing ratio methods}
\section{Calculating temperature}
\subsection{Edge effects}
\subsection{Comparison with alternative methods}
\section{The impact of CO contamination}
\section{The impact of the free-free contribution}
%MWC297
\begin{figure*}
\begin{center}
\includegraphics[scale=0.7]{20140718_MWC297_tempcomp.jpeg}
\caption{Temperature maps of MWC 297 from the ratio of 450\,$\micron$ and 850\,$\micron$ emission pre (\emph{left}) and post (\emph{right}) free-free 
contamination subtraction. Contours are at 11, 25 and 38\,K. The location of MWC 297 is marked with a star.}
\label{fig:Tcomp}
\end{center}
\end{figure*}
%MWC297
Figure~\ref{fig:contamination} presents the 850\,$\micron$ before and after subtraction. Figure~\ref{fig:Tcomp} presents the impact of free-free emission on temperature maps of the region. Even with the free-free emission subtracted, a large, extended submillimetre clump remains, though its peak is offset from the location of MWC 297 by 24.2\arcsec\ (approximately 6,000\,AU).
%MWC 297
The impact of this contamination on the temperature maps is remarkable. The power law of $\alpha = 1.03\pm0.02$ that describes free-free emission from both an UCH\textrm{II} region and jet outflows produces greater flux at 850\,$\micron$ than 450\,$\micron$. Free-free dominates the flux and this results in artificially lower ratios and therefore lower temperatures. This is consistent with the cold spot seen in Figure \ref{fig:Tcomp}a at the location of the UCH\textrm{II} region, with a temperature of approximately 11\,K. We can conclude that free-free emission may contaminate submillimetre temperature maps where cold spots are coincident with hot OB stars.
%MWC 297
The 5$\sigma$ level of 82\,mJy and 11\,mJy means that flux is too uncertain to be detected at 450$\micron$ and therefore it is not possible to calculate reliable temperatures of the residual circumstellar envelope/disk around the star. 
%MWC297
\begin{figure*}
\begin{center}
\includegraphics[scale=0.7]{/Users/damian/Documents/Thesis_et_al/images/20140718_MWC297_tempcomp.pdf}
\caption{Temperature maps of MWC 297 from the ratio of 450\,$\micron$ and 850\,$\micron$ emission pre (\emph{left}) and post (\emph{right}) free-free 
contamination subtraction. Contours are at 11, 25 and 38\,K. The location of MWC 297 is marked with a star.}
\label{fig:Tcomp}
\end{center}
\end{figure*}
%MWC 297
\begin{figure*}
\begin{center}
\includegraphics[scale=0.7]{/Users/damian/Documents/Thesis_et_al/images/20140618_mwc297_contamination.pdf}
\caption{IR1 SCUBA-2 850\,$\micron$ data before \emph{left} and after \emph{right} removal of free-free contamination from an UCH\textrm{II} region and polar jets/winds (represented by the point source contours in the \emph{left} plot). SCUBA-2 contours are at 0.011, 0.022, 0.033 and 0.055 Jy/pixel (corresponding to 5, 10, 15 and 25 $\sigma$ detection limits). 6\,cm VLA contours (red) from Sandell (private comm.) at 0.002, 0.005, 0.02, 0.072, 0.083 Jy/beam are overlaid on the left hand panel. The location of MWC 297 is marked with a star. Beam sizes are shown at the bottom of the image (VLA CnD config. \emph{left} and JCMT \emph{right}.) }
\label{fig:contamination}
\end{center}
\end{figure*}


