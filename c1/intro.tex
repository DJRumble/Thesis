%% Introduction.
% From the university regulations:
%(a) The aims, objectives and results of the candidate's research
%(b) The research methodology where not otherwise described
%(c) The contribution made by the papers in the context of the approved field
%(d) A statement of the candidate's contribution to co-authored papers
%(e) A literature review

\chapter{Introduction}
\label{ch:intro}

\section{Content (to be sorted)}

%%%MWC INTRO MATERIAL %%%%%
The temperature of gas and dust in dense, star-forming clouds is vital in determining whether or not clumps undergo collapse 
and potentially form stars \citep{Jeans:1902dz}. Dense clouds can be heated by a number of mechanisms: heating from 
the interstellar radiation field (ISRF) \citep{Mathis:1983dq, Shirley:2000uq, Shirley:2002vn}, evolved OB stars with HII regions 
\citep{Koenig:2008jo, Deharveng:2012fk} or strong stellar winds \citep{Canto:1984dq, Ziener:1999kl, Malbet:2007zr}; and 
internally through gravitational collapse of the Young Stellar Object (YSO) and accretion onto its surface \citep{calvet98}. 
%Outflows are also observed in the form of jets \citep{Skinner:1993bh, Bontemps:1996fu, Gomez:2004cr} and stellar winds 
%\citep{Drew:1997qf} as a feedback mechanism as well as radiative heating from the photosphere of the star \citep{Hatchell:2013ij}. 
Radiative feedback is thought to play an important role in the formation of the most massive stars through the suppression of 
core fragmentation \citep{Bate:2009uq, Offner:2009pt, Hennebelle:2011ly}. 
%%%MWC INTRO MATERIAL %%%%%
The temperature of star-forming regions has been observed and calculated using a variety of different methods and data. 
Some methods utilise line emission from the clouds: for example, \cite{Ladd:1994ly} and \cite{Curtis:2010zr} examine the 
CO excitation temperature and \cite{Huttemeister:1993ve} looked at a multilevel study of ammonia lines. Often temperature 
assumptions are made in line with models of Jeans instability and Bonnor-Ebert Spheres \citep{Ebert:1955vn, Bonnor:1956vn, 
Johnstone:2000fk}. An alternative method is to fit a single temperature greybody model to an observed Spectral Energy Distribution (SED) 
of dust continuum emission for the YSO \citep{Hildebrand:1983fy}; however, this method is sensitive to the completeness 
of the spectrum, the emission models and local fluctuations in dust properties \citep{Konyves:2010oq, Bontemps:2010fk}.
%%%MWC INTRO MATERIAL %%%%%
Where multiple submillimetre observations exist, low temperatures (less than 20\,K), which favour cloud collapse, can be 
inferred by the relative intensity of longer wavelengths over shorter wavelengths. For example, \emph{Herschel}  provides FIR and 
submillimetre data through PACS bands 70\,$\micron$, 100\,$\micron$ and 160\,$\micron$ and SPIRE bands 250\,$\micron$, 
350\,$\micron$ and 500\,$\micron$ \citep{Pilbratt:2010fk}. \cite{Menshchikov:2010kl, Andre:2010kx} use \emph{Herschel} 
data to construct a low resolution temperature map for the Aquila and Polaris region through fitting a greybody to dust 
continuum fluxes (an opacity-modified blackbody spectrum). \emph{Herschel} data offers five bands of FIR and submillimetre 
observations and low noise levels; however, it lacks the resolution of the JCMT which can study structure on a scale of 
7.9\arcsec (450\,$\micron$) and 13.0\arcsec (850\,$\micron$) \citep{Dempsey:2013uq} as opposed to 25.0\arcsec\ and larger 
for 350\,$\micron$ or greater submillimetre wavelengths. \cite{Sadavoy:2013qf} combine \emph{Herschel} and SCUBA-2 
data to constrain both $\beta$ and temperature.
%%%MWC INTRO MATERIAL %%%%%
This work develops a method which takes the ratio of fluxes at submillimetre wavelengths when insufficient data points exist 
to construct a complete SED. The ratio method allows the constraint of temperature or $\beta$, but not both simultaneously. 
Throughout this paper we used a fixed $\beta$. The value and justification for this are discussed in Section 3. Similar methods 
have been applied by \cite{Wood:1994qf}, \cite{Arce:1999bh} and \cite{Font:2001cr} and used by \cite{Kraemer:2003uf} 
at 12.5\,$\micron$ and 20.6\,$\micron$ and by \cite{Schnee:2005zr} at 60\,$\micron$ and 100\,$\micron$. \cite{Mitchell:2001ve} 
first used 450\,$\micron$ and 850\,$\micron$ fluxes from SCUBA, though full analysis was limited by the quality and quantity of 
450\,$\micron$ data. A more rigorous analysis of SCUBA data was completed by \cite{Reid:2005ly} who are able to constrain 
errors on the temperature maps from sky opacity and the error beam components. Most recently similar methods have 
been used by \cite{Hatchell:2013ij} to analyse heating in NGC1333. This work looks to utilise these methods to further 
investigate radiative feedback in star-forming regions. 

\section{Radiative Transfer Theory}
\subsection{Dust}
\subsection{Dust models/assumptions - opacity}
\subsection{Jayliegh Jeans limit}

\section{Protostars}
\subsection{Jeans mass/stability}
\subsection{SED models}
\subsection{Classification: Alpha}
\subsection{Classification: Bolometeric Luminosity}
\subsection{Classification: Bolometeric temperature}
\subsection{Classification: Colour diagrams}
\subsection{Evolution/timescale: MvsL diagrams}
\subsection{IMF/CMFs}

\section{Radiative feedback}
\subsection{OB stars}
\subsection{Winds}
\subsection{RDI}
\subsection{HII regions}
\subsection{Jets}
\subsection{Radiative heating by OB stars}

\section{Accretion}
\subsection{Discs}
\subsection{Envelopes}

\section{Emission}
\subsection{IR}
\subsection{Submm}
\subsection{Radio}
\subsection{X-ray}

\section{Properties}
\subsection{Flux/photometery}
\subsection{Mass}
\subsection{Column density}
\subsection{Volume density}
\subsection{Proper motion}
\subsection{Extinction}

\section{Molecular Line emission}
\subsection{CO}
\subsection{Masers}
\subsection{Other}

\section{Triggered/none triggered star formation?}

